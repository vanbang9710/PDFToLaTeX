\documentclass[a4paper,12pt]{article}
\usepackage{latexsym}
\usepackage{amsmath}
\usepackage{amssymb}
\usepackage{graphicx}
\usepackage{wrapfig}
\pagestyle{plain}
\usepackage{fancybox}
\usepackage{bm}

\begin{document}

1.2 This is achieveable!

Perhaps remarkably, this last goal is actually achieveable, in a very general way. As we will see in the coming sections, we can start with any algorithm that produces a “point predictor” $\hat{f}_{n}$ that predicts $Y_{n+1}$ from $X_{n+1}$, and turn this into a “set predictor” $\hat{C}_{n}$ that satisfies (1).

The basic idea behind conformal prediction is two-fold. The first key idea can actually be explained in a simpler context, {\it where}{\it there}{\it are}{\it no}{\it features}{\it at}{\it all}, and we just have a sequence $\mathrm{y}_{i}\in \mathbb{R}, i=1$, . . . , $n$ of real-valued response values. Suppose our goal is to find a one-sided prediction interval $\hat{C}_{n}=(-\infty,\hat{q}_{n}$] with

$\mathbb{P}(Y_{n+1}\leq\hat{q}_{n})\geq 1-\alpha.$

(2)

Given this goal (2), a natural place to start would be to set $\hat{q}_{n}$ to be the level $(1-\alpha)$ sample quantile of $Y_{1}$ . , $Y_{n}$, which we denote by

$\hat{q}_{n}=$ Quantile $(1-\displaystyle \alpha;\frac{1}{n}\sum_{i=1}^{n}\delta_{Y_{i}})$ ,

with $\delta_{a}$ denoting apoint mass at $a$, and hence $\displaystyle \frac{1}{n}\sum_{i=1}^{n}\delta_{Y_{i}}$ denoting the empirical distribution of $Y_{1}, \ldots, \mathrm{y}_{n}.$ But this would only give use the approximate result

$\mathbb{P}(Y_{n+1}\leq\hat{q}_{n})\approx 1-\alpha.$

This becomes exact as $ n\rightarrow\infty$, under standard conditions (that ensure convergence of the sample quantile to the population quantile). So can we instead get something that satisfies (2) in finite-sample?

First key idea: use ranks to form adjusted quantiles. This is where the first key idea behind conformal prediction comes in (which in a sense traces back to work on rank-based statistics and permutations by Fisher and Pitman in the $1930\mathrm{s}$). As $Y_{n+1}$ is i.i. $\mathrm{d}$. with $Y_{1}$, . . . , $Y_{n}$, then

the rank of $\mathrm{y}_{n+1}$ is uniformly distributed over the values $Y_{1}$, . . . , $Y_{n+1}.$

(3)

This means that

$\mathbb{P}$ ($Y_{n+1}$ is among the $\lceil(1-\alpha)(n+1)\rceil$ smallest of $Y_{1}$, . . . , $Y_{n+1}$) $\geq 1-\alpha,$

(4)

which is in turn equivalent $\mathrm{to}^{1}$

$\mathbb{P}$ ($Y_{n+1}$ is among the $\lceil(1-\alpha)(n+1)\rceil$ smallest of $Y_{1}$, . . . , $Y_{n}$) $\geq 1-\alpha.$

(5)

The last step is critical: note that we have moved from a comparison between $Y_{n+1}$ and a an order statistic of $Y_{1}$ . , $Y_{n+1}$ in (4) to a comparison between $Y_{n+1}$ and an order statistic of $Y_{1}$, . . . , $Y_{n}$ in (5). This is key, because what is on the right-hand side of the $\leq$sign in (5) is {\it computable}{\it from}{\it just}{\it the} fi{\it rst}$n${\it points}. Accordingly, by defining

$\hat{q}_{n}=\lceil(1-\alpha)(n+1)\rceil$ smallest of $Y_{1}, \ldots, Y_{n}$,   (6)

we have precisely achieved (2).

The formulation in (6) is arguably the most intuitive way to remember how to achieve coverage. There are other equivalent formulations. One such equivalent formulation (we will see more later on) is

$\hat{q}_{n}=$ Quantile $(\displaystyle \frac{\lceil(1-\alpha)(n+1)\rceil}{n};\frac{1}{n}\sum_{i=1}^{n}\delta_{Y_{i}})$ .

(7)

$1\mathrm{To}$ see this, consider the complement of the events (inside the probabilities) in (4), (5). Abbreviate $k=\lceil(1-\alpha)(n+1)\rceil.$

Then $Y_{n+1}>$ the{\it k} smallest of $Y_{1}, \ldots, Y_{n+1}$ is clearly an equivalent statement to $Y_{n+1}>$ the {\it k}smallest of $Y_{1}, \ldots, Y_{n}$, since $Y_{n+1}$ can never be strictly larger than itself. That said, this argument really only makes sense for $k\leq n$, and for $k=n+1,$ which occurs if $\alpha < 1/(n+1)$ , then $\lceil(1-\alpha)(n+1)\rceil=n+1$, then we have to interpret the $(n+1)$ smallest of $\mathrm{y}_{1}, \ldots, \mathrm{y}_{n}$ as being $+\infty$ to equate (4), (5). This is the consistent with interpreting the quantile function in (7) to return $+\infty$ when the input level is $\geq 1.$

2
\end{document}
